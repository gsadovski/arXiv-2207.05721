\documentclass[../../main.tex]{subfiles}

\begin{document}

\section{Introduction}\label{sec:introduction}

General Relativity (GR) is based on A. Einstein's minimalistic assumption that all degrees of freedom of the gravitational field can be encoded in a single tensorial field: the metric field $g$. This historically attracted immediate criticism. Most notably from E. Cartan, who strongly advocated that metrical and affine structures are two logically distinct concepts. In Cartan's point of view, we should minimize \textit{ad hoc} assumption about the spacetime manifold. Thus, the degrees of freedom of the gravitational field should be described, in generality, by $g$ and an affine connection $\Gamma$ as independent fundamental fields.

Today, more than 100 year later, many so-called alternatives to GR have been developed. Most of them falling under the umbrella of metrical (GR, $f\left(R\right)$, \textit{etc.}), scalar-metrical (Brans-Dicke, Horndeski, \textit{etc.}), vector-metrical (Will-Nordtvedt, Hellings-Nordtvedt, \textit{etc.}), bimetrical (Rosen, Rastall, \textit{etc.}), affine (Teleparallel, Symmetric Teleparallel, \textit{etc.}), metric-affine (Einstein-Cartan (EC), metric-affine $f(R)$, \textit{etc.}), or gauge-theoretical (Einstein-Cartan-Sciama-Kibble (ECSK), affine gauge theory, \textit{etc.}). Each of them claiming their own set of fundamental variables.

In teleparallelism~\cite{aldrovandi2013,maluf2013}, there exists a single fundamental field: a metric-compatible flat $\Gamma$. Gravity, in this formulation, is exclusively a manifestation of non-vanishing torsion. In symmetric teleparallelism~\cite{ferraris1982a,nester1999}, it is a torsion-free flat $\Gamma$, and gravity is exclusively manifestation of non-vanishing non-metricity. In EC theory~\cite{hehl1974,hehl1976a,trautman2006}, metric-affine $f(R)$~\cite{sotiriou2007,sotiriou2009,sotiriou2010,olmo2011}, generic metric-affine~\cite{hehl1976b}, \textit{etc.}, $g$ and $\Gamma$ are the two independent fundamental fields (Cartan's philosophy). Gravity, in these, is a manifestation of the non-vanishing curvature, torsion, and/or non-metricity of $\Gamma$.

In the gauge-theoretical branch~\cite{ivanenko1983a,sardanashvily2016a,hehl1995,gronwald1995,gronwald1997,sardanashvily2002a,hehl2012}, $g$ and $\Gamma$ are effective rather than fundamental fields. This approach mainly consists of gauge theories for external symmetries --- usually the general affine group $Aff\left(n\right)$, one of its subgroups, or their supersymmetric extensions --- and, in general, have a soldering form $e$ and a gauge connection $A$ as set of fundamental fields~\cite{hehl1995}. Historically, the Lorentz group $SO\left(1,3\right)$ was the first external group to be gauged, giving raise to the ECSK theory~\cite{utiyama1956,kibble1961,sciama1962,sciama1964}. The generators of $\mathfrak{so}\left(1,3\right)$ are antisymmetric, which makes Lorentz connections metric-compatible. In ECSK theory, gravity is exclusively a manifestation of curvature and/or torsion.

Recently, the gauging of $Spin(4)$ have also shown to produce gravitation~\cite{weldon2001,lippoldt2014,emmrich2022}. In spin base invariant models, the set of fundamental fields consists of Dirac matrices $\gamma$ (in substitution of $e$), satisfying the Clifford algebra $\left\{ \gamma, \gamma \right\} = 2g$, and an $Spin(4)$ gauge connection $\hat A$. Analogously to ECSK theory, the generators of $\mathfrak{spin}(4)$ are antisymmetric, which makes $\hat A$ metric-compatible as well. In spin base invariant models, gravity is a manifestation of non-vanishing curvature and/or torsion of $\hat A$.

As one can see, the \textquote{Einstein \textit{versus} Cartan debate} is pretty much alive~\cite{hehl1974,fay2007,capozziello2011,nojiri2011,clifton2012,berti2015,iosifidis2020,golovnev2022}. And, in despite of the very recent advances in observational physics --- with emphasis in very long baseline interferometry and multi-messenger astronomy ---, a concrete answer seems unlikely in the near future~\cite{lammerzahl1997,obukhov2014,broderick2014,yagi2016,baker2017,mizuno2018,sunny2019,eht2019,bahamonde2021,cantata2021,lobo2021,ferreira2022}. Under this light, a classification of all these distinct description is of upmost importance.

Classically, GR, Teleparallel and Symmetric Teleparallel are known to be equivalent among themselves~\cite{jimenez2019,capozziello2022a}. As we review in more details in Section~\ref{sec:gravities}, EC theory, in the presence of matter carrying vanishing hypermomentum currents, is also equivalent to GR~\cite{hehl1974}. Analogously, metric-affine $f(R)$ in vacuum is equivalent to GR~\cite{sotiriou2007}. This last result can strike as quite surprising, given that the only case in which metrical $f(R)$ is equivalent to GR is when $f(R)=R$~\cite{olmo2007}. If $\phi \equiv f'(R)$ is an invertible field transformation\footnote{The prime $'$ indicates derivative of the function with respect to its argument.}, then metrical $f(R)$ is equivalent to the scalar-metrical $\omega=0$ Brans-Dicke theory with potential $V(\phi) \equiv R(\phi)\phi-f(R(\phi))$~\cite{teyssandier1983}. Under the same field transformation, metric-affine $f(R)$, in the presence of matter carrying vanishing hypermomentum currents, is equivalent to $\omega=-3/2$ Brans-Dicke theory with potential $V(\phi)$~\cite{sotiriou2009}. These $f(R)$ (in){}equivalences smoothly hold in the limit $f(R)\rightarrow R$~\cite{olmo2007}.

Many gauge-theoretical models of gravity have classical equivalence to metrical, affine or metric-affine models. For instance, the gauge theory for spacetime translations ($\mathbb{R}^4$), first developed in~\cite{hayashi1967}, was shown to be equivalent to teleparallelism in~\cite{cho1976,hayashi1977,hayashi1979} --- see~\cite{aldrovandi2013} for a historical account. The ECSK theory, on the other hand, was born as a gauge theory of gravity and is equivalent to EC theory. Spin base invariant gravity (with vanishing spin torsion) and ECSK theory (in the presence of matter with vanishing hypermomentum currents) are both equivalent to GR\@. In fact, these equivalences are so widely spread in physics literature, that they are commonly referred to as just different formalisms of a same underlying physical theory: the metrical or holonomic $(g,\Gamma)$ \textit{versus} the \textit{vielbein} or non-holonomic $(e,A)$\footnote{Or $(\gamma,\hat A)$, in the case of spin base invariant gravity.}.

In this work, we avoid phrasing holonomic $(g,\Gamma)$ \textit{versus} non-holonomic $(e,A)$ theories of gravity as just different formalisms. We do not take the aforementioned equivalences for granted, and we adopt the point of view that holonomic \textit{versus} non-holonomic theories of gravity are fundamentally distinct --- unless unequivocally proven otherwise. The reason for this is three-fold: (i) a classical equivalence might not hold quantum mechanically; (ii) to the author's knowledge, it is not known if this equivalence hold for a general metric-affine dynamics and; (iii) it is known to fail on degenerate spacetimes~\cite{kaul2016a,kaul2016b,kaul2019}.

As one can imagine, point (i) is tricky to be addressed, as we do not know how to properly formulate a quantum theory of gravity. However, attempts have been made within a path integral approach. In~\cite{lippoldt2014}, it was shown that the field transformation given by $\left\{\gamma,\gamma\right\}=2g$, introduces a trivial factor in the functional measure $\mathcal{D}g$ (of quantum GR), if transformed from the functional measure $\mathcal{D}\gamma$ (of quantum spin base invariant gravity with vanishing spin torsion). In~\cite{zanelli2003}, the quantum equivalence between ECSK theory in vacuum and GR is also established. It, however, ignores non-globally hyperbolic spacetimes and the particular ghost structure of each theory\footnote{The former is invariant under $Diff(4) \ltimes SO(1,3)$, while the latter is under $Diff(4)$.}. Reference~\cite{dario2011} shows that such oversights are irrelevant at 1-loop level. Nonetheless, functional renormalization group (FRG) analyses show important qualitative differences in the $g$ \textit{versus} $e$ theory space, and their respective FRG flows~\cite{reuter2010}. More concretely in~\cite{reuter2012}, the ghost fields associated to the local Lorentz invariance are shown to contribute quite significantly to the running of the Newton and cosmological constant in the non-perturbative regime\footnote{This also should not be seen as a definite answer, as the FRG framework relies on a specific truncation and on the use of a Euclidean signature for the metric.}~\cite{reuter2012}. As one can see, the quantum equivalence is pretty much an open issue even in the simplest models --- further discussions can be found in~\cite{reuter2013,reuter2015,reuter2016}.

Point (ii) is addressed in Section~\ref{sec:on-shell_equivalence}, and is the main original contribution of this present work. We establish the equivalence between holonomic \textit{versus} non-holonomic gravity theories, in a manner independent of any particular metric-affine dynamics and/or spacetime dimensions. This equivalence holds as long as the field transformations in~\eqref{eq:field-transformation} hold. In Section~\ref{sec:geometric_picture}, we clarify the global bundle-theoretical nature of these field transformations and their relation to the equivalence principle. It happens that the former is connected to the latter via a non-degenerate $e$. This brings us to point (iii). It refers to regions of spacetime where $e$ or, equivalently, the \textit{vielbein} field $\tensor{e}{^A_\mu}\left(x\right)$ --- its matrix representation --- is non-invertible. As we discuss in more details in Section~\ref{sec:geometric_picture} and~\ref{sec:conclusions}, on these regions, the gauge-theoretical description of gravity unsolders from spacetime, and all the gauge-theoretical equivalences aforementioned are expected to fail.

Sections~\ref{sec:holonomic_and_non-holonomic_frames}~and~\ref{sec:gravities} serve as prelude to the generalizations addressed in Section~\ref{sec:on-shell_equivalence}. They contain a detailed review on holonomic \textit{versus} non-holonomic variables, and the reasons for the classical equivalence among GR, EC and ECSK theory in non-degenerate spacetimes. Finally, in what follows, we consider the category of smooth $n$-dimensional manifolds ($C^\infty$ $n$-manifolds), unless stated otherwise. Greek, lower-case and upper-case Latin letters range from $0$ to $n-1$, unless stated otherwise.

\end{document}
