\documentclass[../../main.tex]{subfiles}

\begin{document}

\section{Conclusions}\label{sec:conclusions}

In the present work, we extended the known classical equivalence between the non-holonomic $S_{\text{VHP}}\left[e,A\right]$ \textit{versus} the holonomic $S_{\text{EHP}}\left[g,\Gamma \right]$ theory of gravity, under field transformations~\eqref{eq:field-transformation}. The equivalence now holds in all spacetime dimensions, and in all metric-affine dynamics --- including arbitrarily high (but finite) high-derivative ones.

We presented a detailed geometric formulation of field transformations~\eqref{eq:field-transformation}, how they encapsulate the equivalence principle, and how their violation might break the equivalences aforementioned. Physically, this break equates to a scenario in which a non-holonomic gauge description of gravity is completely dissociated from spacetime; the internal degrees of freedom are not mimicking the external ones.

A known case in the literature is on degenerate spacetime regions. At them, the vector bundle morphism $e$ is, at most, surjective or injective. The \textit{vielbein} field is non-invertible or, in holonomic language, the metric tensor is singular. In~\cite{kaul2016a,kaul2016b,kaul2019}, it was shown that the classical equivalence between $S_{\text{VEP}}\left[e,A\right]$ and $S_{\text{EHP}}\left[g,\Gamma\right]$ breaks in these regions. Our analysis, in Section~\ref{sec:on-shell_equivalence}, allows to naturally extend this result to the generic metric-affine dynamics. Quantum mechanically, an earlier work by A.~A.~Tseytlin had already noticed this failure once $\det\left(\tensor{e}{^A_\mu}\right)=0$ configurations are allowed in the gravitational path integral~\cite{tseytlin1982}. In general, these spacetime regions are associated to topology-change~\cite{geroch1967,tipler1977,horowitz1991,borde1994,borde1999,heveling2022}.

Another way to violate~\eqref{eq:field-transformation} is to postulate the existence of a non-vanishing tensor field
\begin{equation}
  \label{eq:d-tensor}
  \tensor{D}{^A_{\mu\nu}} \equiv \partial_\mu \tensor{e}{^A_\nu} + \tensor{\Gamma}{^\alpha_{\mu\nu}}\tensor{e}{^A_\alpha}-\tensor{\omega}{^A_{B\mu}}\tensor{e}{^B_\nu}\;.
\end{equation}
In the literature,~\eqref{eq:d-tensor} is sometimes interpreted as the result of applying an exterior covariant derivative, defined on the spliced bundle $TX\times TM$, to $\tensor{e}{^A_\nu}$. Theories with non-vanishing $\tensor{D}{^A_{\mu\nu}}$ must live on the spliced bundle, and are concomitantly holonomic and non-holonomic. Thus, the question of equivalence becomes nonsensical. This scenario, however, presents a novel way to lift spacetime and gauge space symmetries into a single geometrical arena, finding recent applications in 11-dimensional supergravity, higher-spin gravity, and M-theory~\cite{hull2007,engquist2008,nicolai2014a,nicolai2014b}.

\end{document}
