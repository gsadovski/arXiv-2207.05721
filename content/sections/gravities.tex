\documentclass[../../main.tex]{subfiles}

\begin{document}

\section{Gravity theories}\label{sec:gravities}

\subsection{Cartan's philosophy}\label{ssec:cartan's_philosophy}

In holonomic moving frames, a fundamental ingredient of the Einstein-Palatini approach is the understanding that a metric tensor field $g(x)$ and an affine connection $\nabla$ are \textit{\`a priori} two logically distinct concepts over a spacetime region $ U \subseteq X $. Physically, $g(x)$ introduces how a set $ \partial_\mu (x) $ of observers on $ U $, can perform \textquote{dot product} measurements on pairs of vector fields. Indeed, $g\left[\partial_\mu,\partial_\nu\right] \left(x\right)\equiv g_{\mu\nu} \left(x\right)$ is a smooth function on $U$ associating a set of $n\left(n+1\right)/2$ real numbers to each $ x \in U $. Such numbers are interpreted as \textquote{sizes} ($\mu=\nu$) and \textquote{angles} ($\mu\neq \nu$) between $\partial_\mu$ and $\partial_\nu$\footnote{From now on, whenever the context is sufficiently clear, the $x$ dependence of fields will be omitted.}. Meanwhile, $\nabla$ introduces a recipe on how vector fields can be differentiated along others. Indeed, $ dx^\alpha \left( \nabla \left[ \partial_\mu, \partial_\beta \right] \right) \equiv \tensor{\Gamma}{^\alpha_{\beta\mu}} $ is another smooth function on $ U $ associating a set of $n^3$ real numbers for each $ x \in U $, in general. Such numbers are interpreted as the infinitesimal variation of $ \partial_\beta $ along $\partial_\mu$ as measured by $dx^\alpha$ at each $x$.

It is true, however, that there is a canonical way to collapse affine concepts into metrical ones. For instance, to impose that the only way $\partial_\beta$ can vary along $\partial_\mu$ is by an infinitesimal \textquote{rotation}, \textit{i.e.}, a specific first-order change in $g_{\beta\mu}$. Such change is what is known as Christoffel symbols. A detailed analysis of the irreducible decomposition of $\tensor{\Gamma}{^\alpha_{\beta\mu}}$ reveals that this is equivalent to forbid $\partial_\beta$ to pick up any infinitesimal shear, dilatation and/or displacement variations along $\partial_\mu$. Or, equivalently, that $\tensor{\Gamma}{^\alpha_{\beta\mu}}$ satisfies the following constraints:
\begin{subequations}\label{eq:vanishing_torsion_nonmetricity}
  \begin{align}
    \tensor{T}{^\alpha_{\beta\mu}} & \equiv \tensor{\Gamma}{^\alpha_{\beta\mu}}-\tensor{\Gamma}{^\alpha_{\mu\beta}} = 0 \label{eq:vanishing_torsion} \;,                                                       \\
    \tensor{Q}{_{\alpha\beta\mu}}  & \equiv \partial_\alpha g_{\beta\mu} -\tensor{\Gamma}{^\nu_{\beta\alpha}}g_{\nu\mu}-\tensor{\Gamma}{^\nu_{\mu\alpha}}g_{\beta\nu} = 0 \label{eq:vanishing_nonmetricity}\;,
  \end{align}
\end{subequations}
where $\tensor{T}{^\alpha_{\beta\mu}}$ and $Q_{\alpha\beta\mu}$ are the so-called torsion and non-metricity tensor fields, respectively. As one can see, this is indeed a very particular and constrained geometry, first formulated by B. Riemann in 1854\footnote{This work was never published by Riemann himself. He first laid out the foundational aspects of what is now called a Riemannian $n$-manifold in a lecture, \textit{On the hypothesis that lie at the foundation of geometry}, at G\"{o}ttingen University, as part of the qualification process for him to become a \textit{Privatdozent} (lecturer).}, and fully embodied in GR 60 years later\@.

On the other hand, other geometries such as Weitzenb\"{o}ck ($R=0, \; T\neq 0, \; Q=0$), Weyl ($R=0, \; T=0, \; Q\neq 0$), Riemann-Cartan  ($R\neq 0, T\neq 0$, $Q=0$) or metric-affine ($R\neq 0, T\neq 0$, $Q\neq 0$) are as compatible with current observational data as the Riemannian hypothesis~\cite{hehl1974,lammerzahl1997,fay2007,capozziello2011,nojiri2011,clifton2012,obukhov2014,broderick2014,berti2015,yagi2016,baker2017,mizuno2018,sunny2019,eht2019,jimenez2019,iosifidis2020,bahamonde2021,cantata2021,lobo2021,ferreira2022,golovnev2022,capozziello2022a}.\ \textit{Em prol} of generality, and advocating Cartan's philosophy, we consider $g_{\mu\nu}$ and $\tensor{\Gamma}{^\alpha_{\beta\mu}}$ as completely independent fields, each carrying part of the classical degrees of freedom of gravity. Unless, dynamically stated otherwise via the field equations.

Following the above reasoning, the curvature tensor field
\begin{equation}
  \label{eq:curvaturetensor}
  \tensor{R}{^\alpha_{\beta\mu\nu}} = \partial_\mu\tensor{\Gamma}{^\alpha_{\beta\nu}}-\partial_\nu\tensor{\Gamma}{^\alpha_{\beta\mu}}+\tensor{\Gamma}{^\alpha_{\rho\mu}}\tensor{\Gamma}{^\rho_{\beta\nu}}-\tensor{\Gamma}{^\alpha_{\rho\nu}}\tensor{\Gamma}{^\rho_{\beta\mu}} \;
\end{equation}
should be considered as a function of $\tensor{\Gamma}{^\alpha_{\beta\mu}}$ and its derivatives alone --- not of $g_{\mu\nu}$ and its derivatives. This tensor differs from the Riemann curvature since, again, we are not making any assumption on how $\tensor{\Gamma}{^\alpha_{\beta\mu}}$ differs from Christoffel symbols.

\subsection{EC and ECSK theory}\label{ssec:ec_theory}

The dynamics of $n$-dimensional EC theory is defined by the $n$-dimensional Einstein-Hilbert-Palatini (EHP) action,
\begin{equation}\label{eq:holonomic-EP-action}
  S_{\text{EHP}}\left[g_{\mu\nu},\;\tensor{\Gamma}{^\alpha_{\beta\mu}}\right]=\int_{X} d^n x \sqrt{-g}\tensor{R}{^\alpha_{\mu\alpha\nu}}g^{\mu\nu}\;.
\end{equation}
where $g_{\mu\nu}$ corresponds to a Lorentzian metric, while $g$ is its determinant and $g^{\mu\nu}$ is its inverse. The field equations obtained from the functional variation w.r.t. $g_{\mu\nu}$ and $\tensor{\Gamma}{^\alpha_{\beta\mu}}$ are, respectively,
\begin{subequations}\label{eq:holonomic-field-eqs}
  \begin{align}
    -\sqrt{-g}G^{\mu\nu}                                                                                                                                                                                            & = 0 \;, \label{eq:holonomic-einstein-like-field-eqs} \\
    -\sqrt{-g}\left[\tensor{T}{^\mu_\alpha^\beta}-\tensor{Q}{_\alpha^{\beta\mu}} + \frac{1}{2} \left(g^{\beta\mu}\tensor{Q}{_{\alpha\nu}^\nu}+\tensor{\delta}{_\alpha^\mu}\tensor{Q}{^\beta_\nu^\nu}\right) \right] & = 0\;, \label{eq:holonomic-cartan-like-field-eqs}
  \end{align}
\end{subequations}
where $G_{\mu\nu} \equiv \tensor{R}{^\alpha_{\mu\alpha\nu}} - \frac{1}{2} \tensor{R}{^\alpha_{\beta\alpha\lambda}}g^{\beta\lambda}g_{\mu\nu}$ is the post-Riemannian Einstein tensor;  asymmetric in $\mu\nu$.

The $n$-dimensional ECSK theory, which is formulated in non-holonomic frames, has its dynamics defined by the so-called \textit{Vielbein}-Einstein-Palatini (VEP) action,
\begin{equation}
  \label{eq:vep-action}
  S_{\text{VEP}} \left[\tensor{e}{^A_\mu}, \tensor{A}{^A_{B\mu}}\right] = \int_X d^n x e\tensor{F}{^A_{B\mu\nu}}\tensor{e}{^\mu_A}\tensor{e}{^{B\nu}}  \;,
\end{equation}
where $\tensor{e}{^A_\mu}$ is the \textit{vielbein} field, $e$ is its determinant and $\tensor{e}{^\mu_A}$ is its inverse satisfying
\begin{subequations}
  \begin{align}
    \tensor{e}{^A_\mu}\tensor{e}{^\mu_B} & = \tensor{\delta}{^A_B}\;, \label{eq:inversepullbackvielbein} \\
    \tensor{e}{^A_\mu}\tensor{e}{^\nu_A} & = \tensor{\delta}{^\nu_\mu} \label{eq:inversevielbein}\;.
  \end{align}
\end{subequations}
Under a non-holonomic $GL\left(n,\mathbb{R}\right)$ action, they transform as
\begin{subequations}
  \begin{align}
    \label{eq:active_transf_vielbein}
    \tensor{e}{^{A'}_\mu} & = \tensor{\Lambda}{^{A'}_{B}} \tensor{e}{^B_\mu} \;, \\
    \tensor{e}{^\mu_{A'}} & = \tensor{\Lambda}{^{B}_{A'}} \tensor{e}{^\mu_B} \;,
  \end{align}
\end{subequations}
while under a holonomic one, they do as
\begin{subequations}
  \begin{align}
    \label{eq:passive_transf_vielbein}
    \tensor{e}{^A_{\nu'}} & = \tensor{J}{^\mu_{\nu'}} \tensor{e}{^A_\mu} \;, \nonumber \\
    \tensor{e}{^{\nu'}_A} & = \tensor{J}{^{\nu'}_{\mu}} \tensor{e}{^\mu_A} \;.
  \end{align}
\end{subequations}
Furthermore, $\tensor{A}{^A_{B\mu}}$ is a $GL\left(n,\mathbb{R}\right)$ connection and its curvature,
\begin{equation}
  \label{eq:nonholcurvature}
  \tensor{F}{^A_{B\mu\nu}}=\partial_\mu\tensor{A}{^A_{B\nu}}-\partial_\nu\tensor{A}{^A_{B\mu}}+\tensor{A}{^A_{C\mu}}\tensor{A}{^C_{B\nu}}-\tensor{A}{^A_{C\nu}}\tensor{A}{^C_{B\mu}} \;,
\end{equation}
is, unmistakably, a function of connection $\tensor{A}{^A_{B\mu}}$ and its derivatives alone.

The reader might notice that the above fields are being called non-holonomic when they clearly also have holonomic indexes. As we clarify in Section~\ref{sec:geometric_picture}, these fields are local projections on $X$ of truly non-holonomic quantities living on internal bundles. Due to this mixed nature, $g_{\mu\nu}$ and its inverse are still present in the non-holonomic framework --- albeit not as a dynamical field. Additionally, an extra metric, $g_{AB}$, and its inverse $g^{AB}$, are also present in order to \textquote{rise} and \textquote{lower} non-holonomic indexes\footnote{The relation between these two metrics (equation~\eqref{eq:metrics}) is addressed in Section~\ref{sec:geometric_picture}.}.

The field equations obtained from the functional variation w.r.t. $\tensor{e}{^A_\mu}$ and $\tensor{A}{^A_{B\mu}}$ are, respectively,
\begin{subequations}\label{eq:nonholonomic-field-eqs}
  \begin{align}
    e\delta^{\mu\nu\lambda}_{\alpha\beta\gamma}\tensor{e}{^\alpha_A}\tensor{e}{^\beta_B}\tensor{e}{^\gamma_C}\tensor{F}{^{BC}_{\nu\lambda}}                                                                                                          & = 0 \;, \label{eq:nonholonomic-einstein-like-field-eqs} \\
    e \left(\delta^{\mu\nu\lambda}_{\alpha\beta\gamma}\tensor{e}{^\alpha_A}\tensor{e}{^{B\beta}}\tensor{e}{^\gamma_C}\tensor{T}{^C_{\nu\lambda}}-2\delta^{\mu\nu}_{\alpha\beta}\tensor{e}{^\alpha_A}\tensor{e}{^\beta_C}\tensor{Q}{_\nu^{BC}}\right) & = 0 \;, \label{eq:nonholonomic-cartan-like-field-eqs}
  \end{align}
\end{subequations}
where some useful definitions were used\footnote{\setlength{\abovedisplayskip}{-6pt}
  \begin{align*}
    \delta^{\mu_1 \cdots \mu_p}_{\nu_1\cdots \nu_p} & \equiv \frac{1}{\left(n-p\right)!}\epsilon^{\mu_1 \cdots \mu_p \lambda_{p+1}\cdots\lambda_{n}}\epsilon_{\nu_1\cdots\nu_p\lambda_{p+1}\cdots\lambda_n}\;,     \\
    \tensor{T}{^A_{\mu\nu}}                         & \equiv \partial_\mu \tensor{e}{^A_\nu}- \partial_\nu \tensor{e}{^A_\mu} + \tensor{A}{^A_{B\mu}}\tensor{e}{^B_\nu}-\tensor{A}{^A_{B\nu}}\tensor{e}{^B_\mu}\;, \\
    \tensor{Q}{_\mu^{AB}}                           & \equiv \tensor{A}{^{AB}_\mu}+\tensor{A}{^{BA}_\mu} \;.
  \end{align*}}. In particular, $\delta^{\mu_1\cdots \mu_p}_{\nu_1\cdots\nu_p}$ is the generalized Kronecker delta, $\epsilon_{\mu_1\cdots \mu_n}$ is the permutation symbol, $\tensor{T}{^A_{\mu\nu}}$ is the non-holonomic torsion and $\tensor{Q}{_\mu^{AB}}$ is the non-holonomic non-metricity associated to the connection $\tensor{A}{^A_{B\mu}}$. Notice how in non-holonomic frames, it is the vanishing of non-metricity --- rather than torsion --- that is related to symmetries of the connection. Whenever $\tensor{Q}{_\mu^{AB}}=0$, then $\tensor{A}{^{AB}_\mu}=-\tensor{A}{^{BA}_\mu}$. From the symmetry group perspective, this is equivalent to a contraction of $GL(n,\mathbb{R})$ down to one of its (pseudo-)orthogonal subgroups. In our case, $SO\left(1,n-1\right)$. This is, of course, the reason why we blindly did the opposite, in the end of Section~\ref{sec:holonomic_and_non-holonomic_frames}. We expand on that in Section~\ref{sec:geometric_picture}.

\subsection{EC \textit{vs.} ECSK \textit{vs.} GR}\label{ssec:inequivalence}

At first glance, the set of field equations~\eqref{eq:holonomic-field-eqs} seems to deviate from $n$-dimensional Einstein equations. Nonetheless, their physical solutions are still exclusively Einstein $n$-manifolds. This is due to the fact that the EHP action~\eqref{eq:holonomic-EP-action} is explicitly invariant under local projective transformations
\begin{subequations}\label{eq:r-symmetry}
  \begin{align}
    g_{\mu\nu}                          & \rightarrow g_{\mu\nu} \;,                                                              \\
    \tensor{\Gamma}{^\alpha_{\beta\mu}} & \rightarrow \tensor{\Gamma}{^\alpha_{\beta\mu}}+\tensor{\delta}{^\alpha_\beta}U_\mu \;.
  \end{align}
\end{subequations}
This is the so-called $R^d$-symmetry, where $U_\mu$ is an arbitrary vector field~\cite{dadhich2012}. One can show that to choose vanishing $\tensor{Q}{_{\alpha\beta\mu}}$ (and, consequentially, $\tensor{T}{^\alpha_{\beta\mu}}$, via~\eqref{eq:holonomic-cartan-like-field-eqs}) or $\tensor{T}{^\alpha_{\beta\mu}}$ (and, consequentially, $\tensor{Q}{_{\alpha\beta\mu}}$, again, via~\eqref{eq:holonomic-cartan-like-field-eqs}) equates to setting $U_\mu=0$. In other words, these choices are nothing but gauge choices for this projective symmetry. $\tensor{ T }{ ^{ \alpha } _{ \beta \mu } }$ and $ Q_{ \alpha \beta \mu } $ are pure $R^d$-gauge quantities and the traditional EH action --- the action of GR --- is just an $R^d$-gauge fixed version of~\eqref{eq:holonomic-EP-action}. This is true as long as we remain decoupled from matter sources carrying non-vanishing hypermomentum currents. If, for instance, spinorial matter is present, $ \tensor{ T }{ ^{ \alpha } _{ \beta \mu } } $ couples to the spin density tensor and assumes an $R^d$-gauge invariant character. This results in a space of solutions containing Riemann-Cartan $n$-manifolds, and~\eqref{eq:holonomic-EP-action} is an $R^d$-gauge unfixed version of EC theory, not GR\@.

The VEP action~\eqref{eq:vep-action} also enjoys invariance under local projective transformations of the connection, namely,
\begin{subequations}\label{eq:nonholonomic-projective-transformation}
  \begin{align}
    \tensor{e}{^A_\mu}    & \rightarrow \tensor{e}{^A_\mu} \;, \label{eq:projective-transf-vielbein}                                 \\
    \tensor{A}{^A_{B\mu}} & \rightarrow \tensor{A}{^A_{B\mu}}+\tensor{\delta}{^A_B}V_\mu \;, \label{eq:projective-transf-connection}
  \end{align}
\end{subequations}
where $V_\mu$ is an arbitrary vector field. The previous scenario then repeats itself in a non-holonomic fashion. $\tensor{Q}{_\mu^{AB}}$ is a pure $R^d$-gauge quantity proportional to $V_\mu$~\cite{dadhich2012}. In the $R^d$-gauge choice $V_\mu=0$, $\tensor{Q}{_\mu^{AB}}$ and, consequentially, $\tensor{T}{^A_{\mu\nu}}$, via~\eqref{eq:nonholonomic-cartan-like-field-eqs}, vanishes. Again, physical solutions are exclusively Einstein $n$-manifolds as long as there are no couplings to matter sources carrying hypermomentum currents. Otherwise, the solutions are Riemann-Cartan $n$-manifolds and~\eqref{eq:vep-action} is an $ R^{ d } $-gauge unfixed version of ECSK theory.

We just argued that EC and ECSK theory, in non-degenerate spacetime regions and in absence of hypermomentum currents, are both equivalent to GR\@. This was due to the presence of a projective gauge symmetry. Therefore, it is reasonable to expect EC and ECSK to be equivalent to each other \textit{modulus} some gauge artifacts. The non-holonomic to holonomic equivalence is achieved, at the level of field equations and action functionals, by the set of field transformations
\begin{subequations}\label{eq:field-transformation}
  \begin{align}
    g_{\mu\nu}                        & = \tensor{e}{^A_\mu}\tensor{e}{^B_\nu}g_{AB} \;, \label{eq:metrics}                                                                          \\
    \tensor{\Gamma}{^\alpha_{\mu\nu}} & = \tensor{e}{^\alpha_A}\tensor{A}{^A_{B\mu}}\tensor{e}{^B_\nu}-\tensor{e}{^\alpha_A}\partial_\mu\tensor{e}{^A_\nu}\;. \label{eq:connections}
  \end{align}
\end{subequations}
Three points are important to be emphasized about them: (i) the Jacobian matrix of such transformations is clearly not trivial; (ii) it is not even a square matrix; (iii) these transformations assume $\tensor{e}{^\alpha_A}$ exists.

Point (i) is of major importance in the study of this (in)equivalence within a path integral quantization of both theories. It indicates the appearance of non-trivial insertions if one transforms the functional measure from $\mathcal{D}e\mathcal{D}A$ to $\mathcal{D}g\mathcal{D}\Gamma$. Point (ii) reflects the one-to-many nature of the \textquote{inverse} transformations. While a non-holonomic description has a unique holonomic counterpart, a holonomic description has infinitely many non-holonomic versions. All of these versions are $GL\left(n,\mathbb{R}\right)$ gauge transformations of each other --- thus, define only one single physical theory. Finally, point (iii) can be relaxed at the expense of a fixed space topology~\cite{geroch1967,tipler1977,horowitz1991,borde1994,borde1999,heveling2022}, and the equivalence between EC and ECSK\@.

The above facts are well established and can be summarized in the following commutative diagram:

\begin{equation}\label{dia:equivalence}
  \begin{tikzcd}[row sep=huge, column sep=huge]
    \text{\eqref{eq:holonomic-EP-action}} \arrow[d, Rightarrow, "\delta S = 0" description]
    \arrow[r, Leftarrow, "\eqref{eq:field-transformation}"]
    &
    \text{\eqref{eq:vep-action}} \arrow[d, Rightarrow, "\delta S =0" description]
    \\
    \text{\eqref{eq:holonomic-field-eqs}}
    \arrow[r, Leftarrow, "\eqref{eq:field-transformation}"]
    &
    \text{\eqref{eq:nonholonomic-field-eqs}}
  \end{tikzcd} \;. \tag{Diagram 1}
\end{equation}
It is not known in the literature how general this equivalence is. In particular, if it is valid for general metric-affine dynamics --- which lacks projective symmetry. As mentioned in Section~\ref{sec:introduction}, we do expect it to break quantum mechanically. And, classically, we do know~\ref{dia:equivalence} breaks in degenerate spacetime regions. The present work provides a more complete picture of this scenario. In Section~\ref{sec:on-shell_equivalence}, we extend~\ref{dia:equivalence} to very general metric-affine dynamics at any spacetime dimensions. The results make very clear that an invertible \textit{vielbein} is a general requirement for it to hold. Additionally, in Section~\ref{sec:geometric_picture}, we give its geometrical formulation, and clarify its underlying physical meaning.
\end{document}
