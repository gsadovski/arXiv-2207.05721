\documentclass[../main.tex]{subfiles}

\begin{document}
%\begin{titlepage}
\thispagestyle{empty}
%%%%%%%%%
% TITLE %
%%%%%%%%%
{
  \noindent
  \Large
  \bfseries
  About the (in){}equivalence between holonomic \textit{versus} non-holonomic theories of gravity
  \bigskip
  \bigskip
}

%%%%%%%%%%%
% AUTHORS %
%%%%%%%%%%%
%\begin{addmargin}[20mm]{0mm}
{
  \noindent
  \bfseries
  Guilherme Sadovski
  \bigskip
}

%%%%%%%%%%%%%%%
% AFFILIATION %
%%%%%%%%%%%%%%%
{
  \noindent
  \footnotesize
  Okinawa Institute of Science and Technology, 1919--1 Tancha, Onna-son, Kunigami-gun, Okinawa-ken, Japan 904--0495.
  \bigskip
}

%%%%%%%%%%
% EMAILS %
%%%%%%%%%%
{
  \noindent
  \footnotesize
  \rmfamily
  e-mails: \href{mailto:guilherme.sadovski@oist.jp}{guilherme.sadovski@oist.jp}.
  \bigskip
}

%%%%%%%%%%%%
% ABSTRACT %
%%%%%%%%%%%%
{
  \noindent
  \bfseries
  Abstract.\normalfont{} We investigate the scenarios in which a holonomic \textit{versus} a non-holonomic frame description of gravity theories are equivalent. It turns out that classically, the equivalence holds in a way that is independent of the particular dynamics and/or spacetime dimension. This includes general metric-affine dynamics. A global bundle-theoretical investigation is carried out, uncovering the equivalence principle as the culprit. The equivalence holds as long as the equivalence principle holds. This is not something to be expected when non-invertible configurations of the \textit{vielbein} field are taken into account. In such case, the gauge-theoretical description of gravity \textquote{unsolders} from spacetime, and one has to decide if gravity is spacetime geometry or an internal gauge theory.
  \bigskip
}
%\end{addmargin}

%%%%%%%
% TOC %
%%%%%%%
{
  \noindent
  \rule{\textwidth}{1pt}
  \vspace{-4.5ex}
  \tableofcontents
  \noindent
  \rule{\textwidth}{1pt}
}
%\end{titlepage}
\end{document}
